\section{SCOPE}

\begin{slshape}
\color{blue} 
This section is a very top level, simple verbal description of what the item is and where it is used.   See the amplifier example under the \textbf{General} heading.
\bigskip
\end{slshape}


\begin{enumerate}[(a)]
	\item \textbf{\underline{General:}} \textbf{This document describes the design of AndrOS, a minimal operating system. The operating system boots on an amd-64 personal computer, mounts a FAT file system, and accesses files on that file system. This operating system is intended to be an educational project, and not a viable product for many to use.}
\bigskip

\begin{slshape}
\color{blue}
	For a simple module such as the power amplifier module the above description is adequate. 
\end{slshape}
\bigskip


	\item \textbf{\underline{Acronyms:}}\begin{slshape} \color{blue}{I have not chosen to define acronyms because I have a thing about the overuse of acronyms.  You may not be so burdened and find that typing out a name is simply too tedious.  Put those acronyms here.}\end{slshape}
	\item \begin{slshape} \color{blue}{Additional short descriptive paragraphs can be added only if 
	needed for special classification, designation of alternate versions or 
	other material that is part of a top-level description.}\end{slshape}
\end{enumerate}

\begin{slshape}
\color{blue}
\StopSign  Take time to write this section for your project.  Being able to write a simple description of even complex things is a good indicator of how well you understand what you are planning.
\end{slshape}
