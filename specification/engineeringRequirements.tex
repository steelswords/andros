\section{ENGINEERING REQUIREMENTS}
%
%\begin{slshape} 
%	Measurability is a key attribute; without it we have no way of knowing that we have built what we said that we would build.  In the aerospace industry 
%	the process of determining that what we have built meets the plan to which we were building is called verification.  Verification answers the question,
%	``Did we build what we said we would build?"
%	\bigskip
%	
%	Ideally, detail in engineering requirements means that the designer can set off designing and building the system with little or no extra input from the stakeholders.  
%	Since we do not live in an ideal world additional stakeholder input will be required.  When additional input is sought the specification should be revised to reflect clarification
%	or change.  Reality is why we include a Revision History Block at the beginning of the specification.
%	\bigskip
%	
%	An ambiguous requirement, by its very nature, means that I could design and build the system in one, two, or many `wrong' ways.  Since the stakeholder who is paying for this system doesn't want one of the many possible wrong ways we need to ensure that requirements cannot be misinterpreted by the designer. 
% \bigskip
%	
%	
%	
%	Understand:
%		\begin{enumerate}
%			\item ``An engineering requirement is a statement about the system that is unambiguous. There's only one way it can be interpreted, 
%							and the idea is expressed clearly so all of the stakeholders understand it."
%			\item ``An engineering requirement is binding. The customer is willing to pay for it, and will not accept the system without it."
%			\item	``An engineering requirement is testable. It's possible to show that the system either does or does not comply with the statement." (Steve Tockey)
%		\end{enumerate}
%		
%Additionally, requirements must be stated in one place only in this document.  You can reference the requirement in another part of the document by using its paragraph number.  It is often tempting to restate a requirement in another requirement to improve clarity.  However, a requirement stated in two places may only get changed in one place during a revision and possibly  leading to confusion or worse a flawed product.
%\bigskip
%		
%Engineering requirements constitute contacts between between clients and design teams.  Requirements are binding.
%\bigskip		
%		
%Requirements form the base that we design on.  A well written requirement keeps us from ending up with a less than ideal implementation.  
%Defining requirements forces us to think prior to design.
%\bigskip
%		
%The engineer's job at this point is to make the stakeholder requirements of the previous section into measurable, detailed, and unambiguous.  
%Sometimes this process requires that a stakeholder be interviewed again until a proper set of engineering requirements emerge.
%\bigskip 
%
%An old joke in aerospace says that a specification is written and thrown over a high wall to the design group.  The spec. writer then runs away from the wall (laughing).  While formal adoption of this absurd method leads to disaster, we should write specifications as if we could never clarify anything.  The clearer we make the specification now, the lower the cost of change later.
%\bigskip
%
%
%
%\subsection*{\underline{Processing Stakeholder Requirements into Engineering Requirements}}
%
%Stakeholder requirements in the form of user stories are typically poorly organized and indefinite.  A designer cannot design to an indefinite requirement.  How would a designer know if an indefinite requirement was met?  Also, different stakeholders may have requirements that effect overlapping areas of the system but because of the nature of story collection the overlap in these requirements may not be obvious.  There are many more issues and a systematic approach to this processing problem should help. 
%\bigskip
%
%The work of sorting and firming up requirements can be considered tedious.  Engineers are famous for hating tedious work.  It is why we are constantly automating things that we don't want to do.  And when we encounter a tedious task we have a tendency to create a process to limit the amount of tedium that we must endure.  Tedious work reduced to a process becomes less tedious if the process is well designed.  In order to translate the scattered and indefinite stakeholder requirement into organized and definite engineering requirements we are going to follow a two step process:
%
%\begin{enumerate}
%\item Identify and sort stakeholder requirements into fixed categories.
%\item Rewrite the sorted requirements into definitive engineering requirements.
%\end{enumerate}
%
%The details of the process are explained in the following sections.
%
%\subsubsection*{\underline{Identifying and Sorting the Stakeholder Requirements}}
%
%At this point in the process we are faced with the task of identifying and sorting the requirements found in the user stories.  Finding the requirements is pretty easy.  The user wants us to do something and that something is a requirement.  Sorting the requirements is only slightly more difficult because we need bins for the different categories of requirements.  Different companies come up with different bins so I don't feel too bad at coming up with my own set of bins.
%\bigskip
%
%
%
%It helps me to think of the process of translating stakeholder user stories into engineering requirements as a machine with several stages.  We take the often less than definite requirements that we find in user stories, clarify the untranslatable stuff, and sort the results into applicable categories.  The first stage of the process is illustrated in Figure \ref{fig:PreliminarySorting}.
%\bigskip
%
%\begin{figure}[tb]
%\centering
%\includegraphics[angle=0,width=15cm]{PreliminarySorting.pdf}
%\caption{\label{fig:PreliminarySorting} A machine that sorts user stories into appropriate categories.}
%\end{figure}
%
%
%
%\begin{minipage}{\textwidth}
%	We will be sorting the requirements into four categories:
%	\begin{itemize}
%		\item Untranslatable Stuff
%		\item Interface Stuff
%		\item Functional Stuff
%		\item Non-Functional (Support) Stuff
%	\end{itemize}
%\end{minipage}
%
%\bigskip
%
%%These categories shown in the illustration are:
%%
%%\begin{enumerate}
%%\item \colorbox{Goldenrod}{Interface Constraints}
%%\item \colorbox{Lavender}{Method Constraints}
%%\item \colorbox{red}{Functional Requirements}
%%\item \colorbox{green}{Non-Functional Requirements (better called Functional Support Requirements)}
%%\item \colorbox{YellowOrange}{Options}
%%\item (Fluff)
%%\end{enumerate}
%%
%%The colors shown in the above list will be used to highlight pieces of the example stakeholder user stories that correspond to the categories.
%%\bigskip
%
%The categories themselves are intended to cover all aspects of the project in a systematic and an understandable fashion.  Properly chosen categories allow the designer working from this specification to verbally see the system (don't worry, we will also draw a few actual pictures of the system).  I'll explain the categories in the next sections.
%
%\paragraph*{Untranslatable Stuff}
%
%Let's face it we're human and our project stakeholders are human.  Being human we don't always communicate what is either meaningful or useful.  Sometimes, as engineers, when we hear or read a stakeholder statement it may seem vague or more like a wish.  Sometimes the statement may be incredibly obvious, but hard to translate into a measurable requirement.  When we encounter such statements it is up to us to approach the individual stakeholder and ask (politely) what on earth the statement means in terms of the project deliverables.
%\bigskip  
%
%Interacting with our stakeholders early in a project is a good thing.  We establish the lines of communication that we will most definitely need sometime in the project.
%\bigskip
%
%\paragraph*{Interface Stuff}
%
%We design and build things to perform some useful or interesting function.  Interesting things interact or interface with their environment to send information, provide motive power, receive and process information, receive power, fit into an existing slot, etc.  Interfaces make up the useful pieces of a system and are critical to the system.
%\bigskip
%
%The documentation for many complex systems includes a very useful document called an Interface Control Document, or ICD for short.  This document contains all the interface information for the system.  It is a very useful document and helps designers avoid a host of errors.  If such a document is available when the specification document is being written then it is simply referenced in the Applicable Documents Section of the specification.  For our purposes we will define the interface requirements as part of this specification document.
%\bigskip
%
%\paragraph*{Functional Stuff}
%
%The devices we make are intended to do something.  The something that they are supposed to do is defined as functional constraints (designer's hands tied) and functional requirements (designer is free to choose the method).
%\bigskip
%
%
%\paragraph*{Non-Functional (Support) Stuff}
%
%The engineered system is designed to produce certain things and interface in certain ways.  To accomplish its function the system needs a lot of support.  For example, a system is designed to process input data in a specific digital format and output the processed data in a specific digital format.  The format constitute interface constraints, but the connector for the digital formats constitute non-functional or functional support interface constraints.
%\bigskip
%
%The box that keeps the system from getting wet constitutes a non-functional or functional support requirement while the data processing requirements constitute a functional requirement.  Meeting non-functional requirements is as necessary to the system as meeting functional requirements.
%\bigskip
%
%Let's return to the example and start sorting the stakeholder user stories into the four categories.  The stories will be highlighted using four colors with the following meaning:
%\bigskip
%
%\end{slshape}
%
%\begin{enumerate}
%	\item \hilite{Cyan}{Untranslatable Stuff}
%	\item \hilite{Goldenrod}{Interface Stuff}
%	\item \hilite{Magenta}{Requirements Stuff}
%	\item \hilite{Apricot}{Non-Functional Stuff}
%\end{enumerate}
%
%\begin{slshape}
%\color{blue}
%Beginning with the Course Instructor's Pedagogical Requirements.  The marked version is:
%\end{slshape}
%\bigskip
%
%\begin{python}
%TheFile = open('CourseInstructorHandsOn.tex')
%
%lines = TheFile.readlines()
%
%markedLines = [0,3,6,9,12,15,18,21]
%
%hilite = ['Cyan','Cyan','Cyan','Magenta','Magenta','Magenta','Goldenrod','Magenta']
%
%for everyline in markedLines:
%
%    newstring = '\hilite{'+hilite[markedLines.index(everyline)]+'}{'+lines[everyline].rstrip('\n')+'}'
%
%    print(newstring+'\n'+'\\bigskip'+'\n')
%
%TheFile.close()
%\end{python}
%
%\begin{slshape}
%\color{blue}
%
%The Course Instructor's Pedagogical Requirements has one item classified as Interface Stuff dealing with the input impedance of the amplifier.  This item must be further sorted by the machine in Figure \ref{fig:InterfaceSorting}.  This machine determines if the item is a constraint or a requirement.
%
%\end{slshape}
%
%\begin{figure}[tb]
%\centering
%\includegraphics[angle=0,width=15cm]{InterfaceSorting.pdf}
%\caption{\label{fig:InterfaceSorting} A machine that sorts Interface Stuff into appropriate categories.}
%\end{figure}
%
%\begin{figure}[tb]
%\centering
%\includegraphics[angle=0,width=15cm]{FunctionalSorting.pdf}
%\caption{\label{fig:FunctionalSorting} A machine that sorts Functional Stuff into appropriate categories.}
%\end{figure} 
%
%\begin{figure}[tb]
%\centering
%\includegraphics[angle=0,width=15cm]{NonFunctionalSorting.pdf}
%\caption{\label{fig:NonFunctionalSorting} A machine that sorts Non-Functional Stuff into appropriate categories.}
%\end{figure}
%
%
%%
%%
%%
%%
%%
%%\begin{slshape}
%%
%%
%%
%%
%  %
%%
%%
%%\paragraph*{Design Constraints}
%%
%%Constraining a system design amounts to designing part of the system rather than simply specifying it.  Constraint is necessary 
%%for a variety of reasons including but not limited to: reuse of previously designed systems, existing capital equipment,  standard interfaces, and known 
%%environmental conditions.
%%\bigskip
%	%
%%As a rule designers hate constraints.  Designers are a vain bunch.  They see themselves as being the best person for the job and see 
%%constraints as affronts to their competence.  You will save yourself needing to explain a constraint to an affronted designer by 
%%explaining the constraint clearly in the specification (really clearly).
%%\bigskip
%%
%%The sorting categories include two types of constraints:
%%
%%\begin{enumerate}
%%\item Interface Constraints
%%\item Method Constraints
%%\end{enumerate}
%%
%%{\bf{\underline{Interface Constraints}}} exist because the functioning system must interface with something else.  One type of interface is a digital interface such as universal serial bus (USB) and the best way to express this interface constraint is to point to a Referenced Document and say `follow this'.  Analog interfaces are specified with such things as input impedance and voltage/current levels.  Typically simple human/machine interfaces can be specified in an engineering document but if you have to specify a complex human/machine interface there are standards for this type of problem also.  See, for example, the Apple user interface `requirements' are found here (\href{https://developer.apple.com/library/ios/documentation/UserExperience/Conceptual/MobileHIG/}{Apple User Experience Documentation}).  Again, why retype the wheel when you can just reference the wheel specification document.  Referenced documents really are your friend.
%%\bigskip
%%
%%{\bf{\underline{Method Constraints}}} exist because a part of the system has been designed by the spec writer.  As an example a specific part number is stated in the specification and the designer is constrained to use this part to fulfill a requirement.  To reiterate \underline{explain your constraint} or be prepared to do a lot of explaining later.
%%\bigskip
%%
%%\paragraph*{Requirements}
%%
%%I have divided the Requirements into Functional and Non-Functional Requirements.  In my view non-functional requirements are better understood as functional support requirements.
%%\bigskip
%%
% %
%%
%%
%%
%%
%%\subsection*{Processing the Example User Stories}
%%
%%\end{slshape}
%%
%%
%%\subsection*{Requirements and Constraints from the Course Instructor}
%%
%%\begin{itemize}
%	%\item  {\textbf{The course instructor is the primary technical specifier of the controls lab power amplifier.  The device:}}
%	%\begin{enumerate}
%		%\item Must meet the pedagogical requirements of the course (ECE/MAE 5310) for which it is designed.\\[.5cm] \underline{User Story} \\
%		%
%%%\begin{python}
%%%TheFile = open('CourseInstructorHandsOn.tex')
%%%
%%%for eachLine in TheFile:
%    %%print(eachLine, end = '')
%		%%
%%%TheFile.close()
%%%\end{python}
%%\normalfont
%%
%	%\InputIfFileExists{CourseInstructorHandsOn.tex}
%	%
%%
%%
%		       %
%%\bigskip
%%
%		  %
%		%\item Must not present safety/shock hazards to any user.\\[.5cm] \underline{User Story} \\ 
%		%
%%\begin{python}
%%TheFile = open('CourseInstructorSafety.tex')
%%
%%for eachLine in TheFile:
%    %print(eachLine, end = '')
%		%
%%TheFile.close()
%%\end{python}
%		       %
%		%
%		%%As the course instructor, I need a safe module for the students to use.   Safety implies freedom from both electrical and mechanical hazzards.
%		%%\bigskip
%		%%
%		%%The module must be free of sharp edges.
%%%\bigskip
%%
%%% The structure below defines a text segment that I use again in the explanation.  I didn't want to have to change it in many places so it is defined as
%%% a new command that I can typeset anywhere.
%%
%%%
%	%%\global\def\voltageConstraint{The module must not present a dangerous shocking hazard to students.  In order to eliminate a potential dangerous shocking hazard the 
%	%%power amplifier module supply voltages constrained to: $V_{+} \leq 18 \: \text{volts} \: \text{and} \: V_{-} \geq -18 \: \text{volts}$ or a total supply swing of $\leq 36$ Volts.}
%	%
%	%%\voltageConstraint
%	%
%%\bigskip
% 	%
%		%\item Must be robust so that student wiring errors do not damage the module.\\[.5cm] \underline{User Story}\\	
%		%
%%\begin{python}
%%TheFile = open('CourseInstructorWiring.tex')
%%
%%for eachLine in TheFile:
%    %print(eachLine, end = '')
%		%
%%TheFile.close()
%%\end{python}
%%\bigskip
%	%
%	%
%		%\item Must be packaged to meet a specified envelope.\\[.5cm] \underline{User Story} \\
%		%
%%\begin{python}
%%TheFile = open('CourseInstructorTSlot.tex')
%%
%%for eachLine in TheFile:
%    %print(eachLine, end = '')
%		%
%%TheFile.close()
%%\end{python}
%%\bigskip
%%
%		%\item Must have a simple to understand interface that minimizes wiring mistakes.\\[.5cm] \underline{User Story}\\
%		%
%%\begin{python}
%%TheFile = open('CourseInstructorBananaJacks.tex')
%%
%%for eachLine in TheFile:
%    %print(eachLine, end = '')
%		%
%%TheFile.close()
%%\end{python}
%%\bigskip
%		%
%		%
%	%\end{enumerate}
%%
%%\subsection*{Requirements and Constraints from the Laboratory Instructor}
%%\subsection*{Requirements and Constraints from the Laboratory Student}
%%\subsection*{Requirements and Constraints from the ECE Department}
%%
%%\subsubsection*{Finding Interface Constraints}
%%
%%Returning to the user stories in the example.  We can search them for constraints.
%%
%%%\item Must be robust so that student wiring errors do not damage the module.\\[.5cm] \underline{User Story}\\	
%		%
%%\begin{python}
%%TheFile = open('CourseInstructorWiring.tex')
%%
%%for eachLine in TheFile:
%    %eachLine = eachLine.replace('\colorbox{yellow}','')
%    %"""eachLine = eachLine.replace('\colorbox{blue}','')"""
%    %"""eachLine = eachLine.replace('\colorbox{red}','')"""
%    %"""eachLine = eachLine.replace('\colorbox{green}','')"""
%    %print(eachLine, end = '')
%		%
%%TheFile.close()
%%\end{python}
%%\bigskip
%%
%%\end{itemize}
%%
%%
%%
%	%
%%\end{slshape}
%%
%%\subsection{Item Definition}
%%
%%\begin{slshape} \color{blue}
%	%A more detailed description than SCOPE. Good idea to include an illustration. A listing of the major constituent parts or sub-assemblies 
%	%is good here. A reference to the intended use and what the item is a part of goes here.
%	%\bigskip
%	%
%	%You aren't limited to either one specific type of diagram or to a single diagram.  And since there is no perfect diagram you will
%	%have to figure out what diagram or diagrams that will most clearly describe your system.  For example, an embedded system typically
%	%has significant hardware and software requirements and hardware and software system diagrams are appropriate.
%	%\bigskip
%	%
%	%Continuing with the power amplifier module example.
%%\end{slshape}
%%\bigskip
%%
%%\textbf{\underline{System Diagram}}   %(or appropriate system representation)
%%\bigskip
%%
%%\begin{slshape} \color{blue}
%	%All figures need a label centered on the page, one line below the figure. The label must 
%	%follow this format (not the exact words):
%%
%	%\bigskip  
%%
%	%1. System data flow.
%%\end{slshape}
%%
%%\subsection{Interface Definition}
%%
%%
%%\begin{slshape} \color{blue}
%	%All the "\,goes-intas and goes-outas." Further define the items 
%	%covered in the System Diagram.
%%\end{slshape}
%%
%%\begin{enumerate}[(a)]
%	%\item Physical:
%		%\begin{slshape} \color{blue}
%			%Mounting, fluid interface, heat transfer if any, and so forth.  Using the controls lab power amplifier module as an example, the module has a size requirement, a mounting requirement, and a thermal interface with the environment requirement.  These are briefly described below.  Details are deferred at this point.
%		%\end{slshape}
%		%\subsubsection{Module Size}
%		%
%		%\subsubsection{Module Mounting}
%		%
%		%\subsubsection{Module Thermal Interface}
%		%
%		%The module shall interface with the lab environment through a passive heatsink.  Specific thermal requirements are found in paragraph %\ref{thermalManagement}.
%		%\bigskip
%		%
%	%\item Electrical:
%		%\begin{slshape} \color{blue}
%			%Electrical connectors with pin diagrams as illustrations.	
%		%\end{slshape}
%		%
%		%\subsubsection{Connector Requirements}
%		%
%		%\subsubsection{Electrical Interface Requirements}
%		%
%		%\paragraph{Input Voltage Range}
%		%
%		%\subparagraph{} 
%		%The controls lab power amplifier module shall amplify input voltage signals of $\pm 10$ volts within the gain specifications of paragraph \ref{gainSpec}.
%		%
%		%\subparagraph{} 
%		%The controls lab power amplifier shall withstand, without damage, input voltages up the supply rails of the amplifier.
%		%
%		%\paragraph{Output Voltage Range}
%		%
%		%\subparagraph{}
%		%The controls lab power amplifier module shall produce output voltages per the requirements of paragraph \ref{gainSpec}.
%		%
%		%\paragraph{Input Impedance}
%		%
%		%The controls lab power amplifier module shall have an input resistance of no less than $10K\Omega$.
%		%
%		%\paragraph{Output Impedance}
%		%
%		%The controls lab power amplifier module shall appear a resistive load as a ideal voltage source in series with a resistance of $\le 0.01\Omega$.
%		%
%		%
%		%
%	%\item Functional or Informational
%		%\begin{slshape} \color{blue}
%			%What data goes in and out, what does what to what. Data bus 
%			%definition (i.e. MIL-STD-1553 or RS-232, etc.) would go here if needed.
%		%\end{slshape}
%	%
%%\end{enumerate}
%%
%%\begin{slshape} \color{blue}
%	%The following sections are the focal point of the REQUIREMENTS section. The outline may be 
%	%expanded as needed to provide the necessary depth of explanation. Remember to state requirements
%	%in the imperative, "shall".		
%%\end{slshape}
%%
%%
%%%\subsection{Design Constraints}
%%%
%%%
%%%\begin{slshape} \color{blue}
%	%%Constraining a system design amounts to designing part of the system rather than simply specifying it.  Constraint is necessary 
%	%%for a variety of reasons including but not limited to: reuse of previously designed systems, existing capital equipment,  and known 
%	%%environmental conditions.
%	%%\bigskip
%	%%
%	%%As a rule designers hate constraints.  Designers are a vain bunch.  They see themselves as being the best person for the job and see 
%	%%constraints as affronts to their competence.  You will save yourself needing to explain a constraint to an affronted designer by 
%	%%explaining the constraint clearly in the specification (really clearly).
%	%%\bigskip
%	%%
%	%%Continuing with the power amplifier module as an example, we will try to ferret out design constraints from the stakeholder's user stories.
%	%%\bigskip
%	%%
%	%%Use of the word `constraint' in a user's story is a dead give away, but we don't always find the exact word and we will have to work a little harder
%	%%to separate constraints from requirements.
%	%%\bigskip
%	%%
%	%%From the course instructor's User's Story we find the word \underline{constrained} used in the following:
%	%%
%	%%\begin{quote}
%	 %%As a course instructor I need equipment in the control systems lab that provides students with a useful hands-on experience.
\bigskip
  
In order to be useful the lab equipment used must not be mysterious (or overly 'black box').
\bigskip
  
Students need to see the inputs and outputs of the system and understand that they can replicate similar systems in their careers.
\bigskip
		
The power amplifier module must be constructed of a technology that is familiar to both mechanical and electrical/computer engineering students at a late-junior year level.  In order to meet this familiarity requirement the design is constrained to use a power operational amplifier.
\bigskip
		
The power amplifier needs to be a unity voltage gain non-inverting buffer. 
\bigskip

The power amplifier must provide a minimum continuous output current of 8 amps.
\bigskip

The input impedance of the power amplifier must be greater than or equal to 10 kilohms.
\bigskip
		
The largest time constant of the power amplifier must be less than 0.001 seconds.
\bigskip
%	%%\end{quote}
%	%%
%	%%We can now safely write our first design constraint.
%	%%
%%%\end{slshape}
%%
%%\subsubsection{Amplifying Element Constraint}
%%
%%The controls lab power amplifier module shall use a commercially available power operational amplifier as it amplifying element.  This constraint is placed on the module design
%%because of mechanical and electrical/computer engineering student familiarity with operational amplifiers.
%%\bigskip
%%
%%\begin{slshape} \color{blue}
%%Continuing with the course instructor's user story, we look for other instances of words similar to constraint.
%%\bigskip  
%%
%%It didn't take long to find the next one.
%%
%%\begin{quote}
%	%As the course instructor, I need a safe module for the students to use.   Safety implies freedom from both electrical and mechanical hazards.
\bigskip
		
The module must be free of sharp edges.
\bigskip
		
The module must not present a dangerous shocking hazard to students.
\bigskip

In order to eliminate a potential dangerous shocking hazard the power amplifier module supply voltages are constrained to: $V_{+} \leq 18 \: \text{volts} \: \text{and} \: V_{-} \geq -18 \: \text{volts}$ or a total supply swing of $\leq 36$ Volts.
\bigskip
%%\end{quote}
%%
%%This constraint is troublesome.  While the course instructor really doesn't want to shock students (a rare instructor indeed), the designer of this module has little control over the power supply voltages.  The responsibility for these voltages is likely fixed by available equipment.  The course instructor would have been better off stating that, due to available lab equipment, the power supply voltages are limited to a maximum of $\pm V$.  An even better constraint would set a range of acceptable voltages such as 12 to 18 volts. Let's write this constraint in that way.
%%\bigskip 
%%
%%\end{slshape}
%%
%%\subsubsection{Power Supply Voltage Constraint}
%%
%%The control power operational amplifiers shall operate on power supplies with output voltages symmetric about a ground reference voltages $\lvert {V_{+}} \rvert = \lvert {V_{-}} \rvert = V$  with $V$ in the range of $10 \:\text{volts} \le V \le 18 \: \text{volts}$.  The power supply voltages of the power amplifier module are constrained to these value due to the availability of laboratory power supplies that operate in this range.
%%\bigskip
%%
%%\begin{slshape}
%%\color{blue}
%%
%%Continuing the search for constraints in the stakeholder's user stories.  The next course instructor comment mentions:
%%\end{slshape}
%%
\subsection{Interface}

\subsubsection{Interface Method Design Constraints}
\paragraph{Hardware Requirements} The OS shall run on an amd64 architecture with at least 256 MiB of RAM.

\subsubsection{Interface Design Requirements}
	\paragraph{Shell Requirements}
	The shell is the main or only way the user has to interact with the OS. It shall receive text input from the user and parse it into commands. It shall support the following operations:
  \begin{enumerate}
    \item Handling multiple command line arguments to a program,
    \item Piping output from one command to the input of another command,
    \item Piping output from one command to an output file, and
		\item Piping an input file as input to a command.
  \end{enumerate}
	

%\subsubsection{Interface Support Design Constraints}

%\subsubsection{Interface Support Design Requirements}



\subsection{Functional Requirements}

%\subsubsection{Functional Method Design Constraints}
%There are no functional method design constraints.

\subsubsection{Functional Design Requirements}
  \paragraph{Booting}
  \begin{enumerate}
    \item The OS shall boot and run from a FAT-formatted drive.
    \item The OS shall boot to a shell in no more than 60 seconds on the reference computer and accept user input in that shell.
    \item The OS shall be compatible with GRUB 2 and allow other operating systems installed to the system to boot from a GRUB menu as well.  
  \end{enumerate}
  \paragraph{File System Actions}
  \begin{enumerate}
    \item Once booted, the OS shall make the following actions available to the user:
    \begin{enumerate}[(a)]
      \item List all mountable FAT-formated devices,
      \item Mount a mountable FAT-formated devices, and
      \item Unmount a mounted FAT-formated device.
    \end{enumerate}
    \item The OS's shell or related utility programs shall let the user do the
      following within a mounted FAT file system:
    \begin{enumerate}[(a)]
      \item List the contents of the file system,
      \item List the contents of a given directory,
      \item Traverse directories,
      \item Print the contents of any file,
      \item Edit the content of any file,
      \item Create directories,
      \item Remove files, and
      \item Remove directories.
    \end{enumerate}
  \end{enumerate}
  \paragraph{Program Loading and Execution}
  \begin{enumerate}
    \item The OS's shell and related utilities shall have the capability to load and execute flat binary executables. While linked libraries are a nice feature, they they are not required for AndrOS.
  \end{enumerate}




%\subsection{Support Requirements}

%\subsubsection{Support Method Design Constraints}

%\subsubsection{Support Design Requirements} 

